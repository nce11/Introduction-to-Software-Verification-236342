\documentclass{article}
% basics
\usepackage{amsfonts}
\usepackage{enumitem}
\usepackage{float}
\usepackage{graphicx}
\usepackage{hyperref} 
\usepackage[labelfont=bf]{caption}

\newtheorem{theorem}{Theorem}
\newtheorem{lemma}[theorem]{Lemma}
\newtheorem{corollary}{Corollary}[theorem]



% unique math expressions:  
\usepackage{amsmath}
\DeclareMathOperator*{\andloop}{\wedge}
\DeclareMathOperator*{\pr}{Pr}
\DeclareMathOperator*{\approach}{\longrightarrow}
\DeclareMathOperator*{\eq}{=}

% grey paper
\usepackage{xcolor}
% \pagecolor[rgb]{0.11,0.11,0.11}
% \color{white}

% embedded code sections
\usepackage{listings}
\definecolor{codegreen}{rgb}{0,0.6,0}
\definecolor{codegray}{rgb}{0.5,0.5,0.5}
\definecolor{codepurple}{rgb}{0.58,0,0.82}
\lstdefinestyle{mystyle}{
    commentstyle=\color{codegreen},
    keywordstyle=\color{magenta},
    numberstyle=\tiny\color{codegray},
    stringstyle=\color{codepurple},
    basicstyle=\ttfamily\footnotesize,
    breakatwhitespace=false,         
    breaklines=true,                 
    captionpos=b,                    
    keepspaces=true,                 
    numbers=left,                    
    numbersep=5pt,                  
    showspaces=false,                
    showstringspaces=false,
    showtabs=false,                  
    tabsize=2
}

\lstset{style=mystyle}

\begin{document}
\author{Yosef Goren, Andrew Elashkin}
\date{November 2022}
\title{Introduction to Software Verification 236342, Homework 1}
\maketitle
\section*{Question 1}
\subsection*{a}
%enumerate with capital letters:
For the program P prove the following using Floyd's proof system:
$$
\{x=X \wedge y=Y\} P\left\{z=X^Y\right\}
$$
We choose 3 cutpoints in the program: $l_0, l_2 \text{ and } l_*$. The inductive assertions for those points are:
$$ I_{l_0} = q_1 = (x=X \wedge y=Y)$$
$$ I_{l_2}=(X^Y= z*x^y) $$
$$ I_{l_*} = q_2 =(z=X^Y) $$
Now, For every basic path $\alpha=\left(l, l^{\prime}\right)$ we need to prove that for $\forall x$: 

\begin{align}
q_1(\bar{x}) \wedge R_{l_0 l_2}(\bar{x}) &\rightarrow I\left(T_{l_0 l_2}(\bar{x})\right) \\
I(\bar{x}) \wedge R_{l_2 l_2}(\bar{x}) &\rightarrow I\left(T_{l_2 l_2}(\bar{x})\right) \\
I(\bar{x}) \wedge R_{l_2 l_*}(\bar{x}) &\rightarrow q_2\left(T_{l_2 l_*}(\bar{x})\right)
\end{align}

With this split we have 4 basic paths in our program flowchart, let us go through all of them one by one.

\begin{enumerate}


 \item path $\alpha_1: l_0 \rightarrow l_1 \rightarrow l_2$ 

$R_\alpha^2(x, y, z)=$ true $\rightarrow$
$R_\alpha^1(x, y, z)=$ true $\rightarrow$
$R_\alpha(x, y, z)=$ true 

$T_\alpha^2(x, y, z)=(x, y, z)$ $\rightarrow$
$T_a^1(x, y, z)=(x, y, 1)$ $\rightarrow$
$T_\alpha(x, y, z)=(x, y, 1)$ \\

Now we check the inductive assertion for $\alpha_1$: \\
$\forall(x, y, z)\left[((x=X \wedge y=Y) \wedge\right.$true$\left.) \rightarrow\left(X^Y=z*x^y \right)[(x, y, z) \leftarrow(x, y, 1)]\right] \equiv$ \\
$\equiv \forall(x, y, z) \left[((x=X \wedge y=Y) \wedge\right.$true$\left.) \rightarrow\left(X^Y=x^y \right) \right]$ and it holds.
\\

 \item path $\alpha_2: l_2 \rightarrow l_*$ \\
$R_{\alpha_2}^1(x, y, z)=$ true $\rightarrow$ 
$R_{\alpha_2}(x, y, z)= (y=0)$  \\
$T_{\alpha_2}^1(x, y, z)=(x, y, z)$ $\rightarrow$
$T_{\alpha_2}(x, y, z)=(x, y, z)$ \\

Now we check the inductive assertion $\alpha_2$: \\
$\forall(x, y, z)\left[\left(\left(X^Y=z*x^y \right)\right.\right.\left.\wedge(y=0)) \rightarrow\left(z=X^Y \right)[(x, y, z) \leftarrow(x, y, z)]\right] \equiv$ \\
$\equiv \forall(x, y, z)\left[\left(X^Y=z*x^0 \right) \rightarrow\left(X^Y=z \right) \right]$ and it holds.

% 3333333333

 \item  path $\alpha_3: l_2 \rightarrow l_3 \rightarrow l_4 \rightarrow l_2$ 

$R_\alpha^3(x, y, z)=$ true $\rightarrow R_\alpha^2(x, y, z)=$ true $\rightarrow$ 

$R_\alpha^1(x, y, z)= \neg \text{even}(y)$ $\rightarrow$
$R_\alpha(x, y, z)= \neg \text{even}(y) \wedge \neg (y=0)$

\vspace{5mm}

$T_\alpha^3(x, y, z)=(x, y, z)$ $\rightarrow
T_\alpha^2(x, y, z)=(x,y-1,z*x)$ $\rightarrow$

$T_a^1(x, y, z)=(x,y-1,z*x)$ $\rightarrow
T_\alpha(x, y, z)=(x,y-1,z*x)$ \\

Now we check the inductive assertion for $\alpha_3$: \\
$\forall(x, y, z)\left[(X^Y=z*x^y) \wedge (\neg \text{even}(y) \wedge \neg (y=0)) \rightarrow\left(X^Y=z*x^y \right)[(x, y, z) \leftarrow(x,y-1,z*x)]\right] \equiv$ \\
$\forall(x, y, z)\left[(X^Y=z*x^y) \wedge (\neg \text{even}(y) \wedge \neg (y=0)) \rightarrow (X^Y=(z*x)*x^{y-1}) \right]$ and it holds since $(z*x)*x^{y-1}=z*x^y$.
\\

% 44444444444

\item  path $\alpha_4: l_2 \rightarrow l_3 \rightarrow l_5 \rightarrow l_2$ 

$R_\alpha^3(x, y, z)=$ true $\rightarrow R_\alpha^2(x, y, z)=$ true $\rightarrow$ 

$R_\alpha^1(x, y, z)=  \text{even}(y)$ $\rightarrow$
$R_\alpha(x, y, z)= \text{even}(y) \wedge \neg (y=0)$ \\

$T_\alpha^3(x, y, z)=(x, y, z)$ $\rightarrow
T_\alpha^2(x,y,z)=(x*x, y/2, z)$ $\rightarrow$

$T_a^1(x, y, z)=(x*x, y/2, z)$ $\rightarrow
T_\alpha(x, y, z)=(x*x, y/2, z)$ \\

Now we check the inductive assertion for $\alpha_4$: \\
$\forall(x, y, z)\left[(X^Y=z*x^y) \wedge ( \text{even}(y) \wedge \neg (y=0)) \rightarrow (X^Y=z*x^y) [(x, y, z) \leftarrow(x*x, y/2, z)]\right] \equiv$ \\
$\forall(x, y, z)\left[(X^Y=z*x^y) \wedge (\neg \text{even}(y) \wedge \neg (y=0)) \rightarrow (X^Y=z*(x^2)^{y/2}) \right]$ and it holds since $z*(x^2)^{y/2}=z*x^y$.
\\
\end{enumerate}

\subsection*{b}

For the program P prove the following using Floyd's proof system:
$$
\langle x=X \wedge y=Y \wedge Y \geq 0\rangle P \langle z=X^Y\rangle
$$
In this question we will use the same cutpoints, the same calculations for $R_\alpha$ and $T_\alpha$ that we had in (a) and a well founded set $(\mathbb{N}, <)$ over natural numbers. The new inductive assertions will be:

$$ I_{l_0} = ((w=y+2) \wedge(x=X \wedge y=Y))$$
$$ I_{l_2}=(w=y+1) \wedge (X^Y= z*x^y)) $$
$$ I_{l_*} = ((w=y) \wedge(z=X^Y)) $$
(INIT) $$
\forall(x, y, z)[(x=X \wedge y=Y \wedge Y \geq 0) \rightarrow \exists w[(w=y+2) \wedge(x=X \wedge y=Y)]]
$$
(DEC)
$$
\forall(x,y,z,w)\left[I_l(x,y,z, w) \wedge R_\alpha(x,y,z) \rightarrow \exists w^{\prime}\left[w^{\prime}<w \wedge I_{l^{\prime}}\left(T_\alpha(x,y,z), w^{\prime}\right)\right]\right]
$$

Now we need to check if those conditions hold for the paths from (a)
\begin{enumerate}


 \item path $\alpha_1: l_0 \rightarrow l_1 \rightarrow l_2$ 
 
 $\forall(x, y, z, w)[ ((w=y+2) \wedge(x=X \wedge y=Y) \wedge $true$ \rightarrow $ \\
 $\exists w^{\prime}[w^{\prime}<w \wedge((w^{\prime}=y+1) \wedge (X^Y=z*x^y)[(x, y, z) \leftarrow(x, y, 1)]]  \equiv$ \\
$\equiv \forall(x, y, z) \left[((w=y+2) \wedge (x=X \wedge y=Y) \wedge\right.$true$\left.) \rightarrow\left(X^Y=x^y \right) \right]$

 \item path $\alpha_2: l_2 \rightarrow l_*$
 
  $\forall(x, y, z, w)[ ((w=y+1) \wedge (X^Y= z*x^y)) \wedge $(y=0)$ \rightarrow $ \\
 $\exists w^{\prime}[w^{\prime}<w \wedge((w^{\prime}=0) \wedge (X^Y=z)[(x, y, z) \leftarrow(x, y, z)]]  \equiv$ \\
$\equiv \forall(x, y, z) \left[((w=1) \wedge (z=X^Y) \rightarrow \exists w^{\prime}[w^{\prime}<w \wedge((w^{\prime}=0) \wedge \left(X^Y=z \right) \right]$
 
  \item  path $\alpha_3: l_2 \rightarrow l_3 \rightarrow l_4 \rightarrow l_2$ 
  
$\forall(x, y, z, w)[ ((w=y+1) \wedge (X^Y= z*x^y)) \wedge (\neg \text{even}(y) \wedge \neg (y=0)) \rightarrow\left(X^Y=z*x^y)[(x, y, z) \leftarrow(x,y-1,z*x)]\right] \equiv$ \\
$\exists w^{\prime}[w^{\prime}<w \wedge((w^{\prime}=y+1) \wedge (X^Y=z*x^y)[(x, y, z) \leftarrow(x, y-1, z*x)]]  \equiv$

$\forall(x, y, z, w)[ ((w=y+1) \wedge (X^Y= z*x^y)) \wedge (\neg \text{even}(y) \wedge \neg (y=0)) \rightarrow\left(X^Y=z*x^y)[(x, y, z) \leftarrow(x,y-1,z*x)]\right] \equiv$ \\
$\exists w^{\prime}[w^{\prime}<w \wedge((w^{\prime}=y+1) \wedge (X^Y=x*z*x^{y-1})]$

and since $x*z*x^{y-1} = z*x^y$ it is true.

 
 \item  path $\alpha_4: l_2 \rightarrow l_3 \rightarrow l_5 \rightarrow l_2$ 

$\forall(x, y, z, w)[ ((w=y+1) \wedge (X^Y= z*x^y)) \wedge ( \text{even}(y) \wedge \neg (y=0)) \rightarrow\left(X^Y=z*x^y)[(x, y, z) \leftarrow(x,y-1,z*x)]\right] \equiv$ \\
$\exists w^{\prime}[w^{\prime}<w \wedge((w^{\prime}=y+1) \wedge (X^Y=z*x^y)[(x, y, z) \leftarrow(x*x, y/2, z)]]  \equiv$

$\forall(x, y, z, w)[ ((w=y+1) \wedge (X^Y= z*x^y)) \wedge (\text{even}(y) \wedge \neg (y=0)) \rightarrow\left(X^Y=z*x^y)[(x, y, z) \leftarrow(x,y-1,z*x)]\right] \equiv$ \\
$\exists w^{\prime}[w^{\prime}<w \wedge((w^{\prime}=y/2+1) \wedge (X^Y=z*(x^2)^{y/2})]$
 
and since $z*(x^2)^{y/2}=z*x^y$ it is true.

\end{enumerate}



\section*{Question 2}
\subsection*{2.a}
Let $P\in FPG'$ (extended Flowchart Programming Languge - with interrupts).\\
And let $CFG:FPL\longrightarrow [n]\times [n]^2$ be a function returning
the control flow graph any program $P\in FPG$ (unmodified).\\

Define $CFG':FPL'\longrightarrow [n]\times [n]^2$ on input $(P,P_{int})$ to be the graph
$CFG(P)$ with the following modifications:
Any node $v\in P$ now has an additional output
goes into the initial
state of a private copy of $CFG(P_{int})$
(each $v\in P$ has it's own copy); Denote this private
copy with $S_v$. Additionally, all final states of $S_v$ go 
into a new node $F_v$, which goes back into $v$.\\

Define a reachability condition $R$ and state transformation $T$ for any path $\tau$ of $FPL'$ program graphs as follows:
\begin{itemize}
    \item Denote $\tau = l_{i_0},l_{i_1},...,l_{i_k}$.
    \item Denote the state transformation of $P_{int}$ (from it's initial to final state - or the one appended to it to be exact) with $T_{int}$.
    \item Let:
    \[
        Option_{int}(S)=S_\tau(\bar{x})\cup
            \{T_{int}(s) \mid s\in S_\tau(\bar{x})\wedge q_{int}(s)\}
    \]
    Inituitively, this transformation expands the set of states possible
    by possibly applying the state transformation of $P_{int}$ to all states.
    \item Base case:
    \[
        T^k_\tau(\bar{x})=\{(\bar{x})\}, R^k_\tau(\bar{x})=true
    \]
    Note how now the transformation function outputs a set rather than a single state.
    This is meant to capture the indetermenism of the interrupt; it represents
    the set of all possible states that can result from an initial (input) state.
    \item Inductive case: Given the functions have been defined for $l_{i_{m+1}}$,
    define them over what is at label $l_{i_m}$:
    \begin{itemize}
        \item \emph{start} or \emph{end}:
        \[
            T^m_\tau(\bar{x})=Option_{int}(T^{m+1}_\tau(\bar{x}))
        \]\[
            R^m_\tau(\bar{x})=R^{m+1}_\tau(\bar{x})
        \]
        \item $\bar{x}:=\bar{y}$:\\
        WLOG we can assume all assignments are to all variables - when this is not the case, we can append identity assignments.
        \[
            T^m_\tau(\bar{x})=Option_{int}(T^{m+1}_\tau(\bar{y}))    
        \]
        \[
            R^m_\tau(\bar{x})=R^{m+1}_\tau(\bar{y})
        \]
        \item $B(\bar{x})$ (boolean branch expression):
        \[
            T^m_\tau(\bar{x})=Option_{int}(T^{m+1}_\tau(\bar{x}))
        \]\[
            R^m_\tau(\bar{x})=
            \begin{cases}
                R^{m+1}_\tau(\bar{x})\wedge[\exists s\in B(T^{m+1}_\tau(\bar{x}))] & \text{if } l_{i_m}\rightarrow^T l_{i_{m+1}}\\
                R^{m+1}_\tau(\bar{x})\wedge[\exists s\in \neg B(T^{m+1}_\tau(\bar{x}))] & \text{if } l_{i_m}\rightarrow^F l_{i_{m+1}}
            \end{cases}
        \]
        Inituitively; we require to get to the conditional label, and then have the
        boolean condition satisfiable by one of the possible states.

        
    \end{itemize}
\end{itemize}


Now we define a modified floyed proof system which follows the steps
on input $(P,P_{int})$:
\begin{enumerate}
    \item Choose a set of cut points s.t.:
    \begin{itemize}
        \item The set contains all initial and final states.
        \item Every cycle in the graph $CFG'(P,P_{int})$ contains at least one cut point.
        \item For every cut point $l$ find an inductive assertion $I_l(\bar{x})$.
            Additionally it is required that: $I_{l_0}(\bar{x}) = q_1(\bar{x}), I_{l_*}(\bar{x}) = q_2(\bar{x})$ where $l_0$
            is the initial state and $l_*$ is a terminal state.
        \item For every simple path between two cut points $l_{i_m}, l_{i_j}$,
        \[
            [I_{l_{i_m}}(\bar{x})\wedge R^m_\tau(\bar{x})\rightarrow I_{l_{i_j}}(\bar{x})]
        \]
    \end{itemize}
\end{enumerate}

The proof system is sound similarly to the original one; 
after the conditions have been shown - we know that for any path from
$l_0$ to $l_*$, $I_{l_0}(\bar{x})\rightarrow I_{l_*}(\bar{x})$ from closure on transitivity
of  the cut points conditions.\\
Since the initial and final conditions are the same as $q_1(\bar{x})$ and $q_1(\bar{x})$ -
we have '$\{q_1(\bar{x})\}(P,P_{int})q_1(\bar{x})$'.

The new proof system is also reasonably complete, as haveing an interrupt which an [\emph{false}]
predicate would mean our proof system is equivalent to the original one.

\subsection*{2.b}
The new proof system we define will be identical to the original floyed's system,
with the following modifications:
\begin{enumerate}
    \item The set of cut points must be the set of all vertecies in the $CFG$.
    \item The new set of conditions is:\\
    For any simple path $\tau$ from two cut points $l_{i_u}, l_{i_v}$ the following formula is satisfied:
    \[
        [I_{i_u}(\bar{x})\wedge R_\tau^{i_u}(\bar{x})\rightarrow \neg q_{int}(T_\tau^{i_u}(\bar{x}))\wedge I_{i_v}(\bar{x})]
    \]
    \item It is still required that $I_{l_0}=q_1$ (the precondition),
    but it is no longer required that $I_{l_*}=q_2$ (the postcondition).
\end{enumerate}

The soundness of the proof system is due to the fact that the closure proprety
will yield every cut point (thus every label) that is reachable to not satisfy the
interrupt condition - meaning it is never possible to enter a run within any sate that can actually be reached.

It is also reasonably complete since we have to require
every reachable state to not satisfy the interrupt condition.


\section*{Question 3}
Let a \emph{'critical'} be a point from which every proceeding point satisfies $q_2$.\\

The idea is to construct two sets of formulas which will each apply to the entier $CFG$,
one will ensure that a \emph{'critical'} point will be reached in the future (this can be done
using well founded sets as seen in the lecture),
and other will ensure that if the \emph{'critical'} point is reached -
the current point satisfies $q_2$ and the next point is also \emph{'critical'}.



Formally, the proof system will require the following steps:
\begin{enumerate}
    \item Define all vertecies in the $CFG$ to be cut points.
    \item Choose a well founded set $(W,<)$.
    \item For each cut point $l$, define two formulas $Pre_l(\bar{x},w)$ and $Post_l(\bar{x})$ where
    the first one depends on the state $\bar{x}$ and an item $w$ from the well founded set - while the second
    only depends on the state.
    \item Require this ('initial') formula to be satisfied:
    \[
        \forall\left[q_1(\bar{x})\rightarrow \exists w: Pre_{l_0}(\bar{x},w)\right]
    \]
    \item  At every simple path $\tau$ between cut points $l_{i_u}, l_{i_v}$,
    require (all) the following formulas be satisfied:
    \begin{enumerate}
        \item \[
            \forall\bar{x}, w
            \left[
                Pre_{l_{i_u}}((\bar{x},w)\wedge R_\tau(\bar{x}))\rightarrow
                ((\exists w'<w: Pre_{l_{i_v}}(\bar{x},w'))\vee Post_{l_{i_v}}(\bar{x}))
            \right]
        \]
        \item \[
            \left[Post_{l_u}(\bar{x})\rightarrow Post_{l_v}(\bar{x})\right]
        \]
        \item \[
            \left[Post_{l_v}(\bar{x})\rightarrow q_2(\bar{x})\right]    
        \]
    \end{enumerate}
    Here in (a) we ensure that from any reachable point,
    we advance to a 'lower' point in the well founded set while maintaiing
    our set of formulas - or otherwise - reach the \emph{'critical'} point.\\
    In (b) we ensure that once we reach the \emph{'critical'} point,
    we maintain the \emph{'critical'} property.\\
    Finally, (c) ties the critical property to the wanted \emph{'postcondition'} $q_2$.
\end{enumerate}

Here we have a sound proof system, since we have ensured that:
\begin{enumerate}
    \item If the precondition is met - we will reach a \emph{'critical'} point.
    \item If we reach a \emph{'critical'} point - we will maintain the \emph{'critical'} property.
    \item If we are passed a \emph{'critical'} point - we satisfy the \emph{'postcondition'} $q_2$.
\end{enumerate}
If we wanted to prove this formally, we could follow a similar proof to the one shown in class
of the proof system for showing termination properties using well founded sets.

\end{document}
