\documentclass{article}
% basics
\usepackage{amsfonts}
\usepackage{enumitem}
\usepackage{float}
\usepackage{graphicx}
\usepackage{hyperref} 
\usepackage[labelfont=bf]{caption}

\newtheorem{theorem}{Theorem}
\newtheorem{lemma}[theorem]{Lemma}
\newtheorem{corollary}{Corollary}[theorem]

% unique math expressions:  
\usepackage{amsmath}
\DeclareMathOperator*{\andloop}{\wedge}
\DeclareMathOperator*{\pr}{Pr}
\DeclareMathOperator*{\approach}{\longrightarrow}
\DeclareMathOperator*{\eq}{=}

% grey paper
\usepackage{xcolor}
% \pagecolor[rgb]{0.11,0.11,0.11}
% \color{white}

% embedded code sections
\usepackage{listings}
\definecolor{codegreen}{rgb}{0,0.6,0}
\definecolor{codegray}{rgb}{0.5,0.5,0.5}
\definecolor{codepurple}{rgb}{0.58,0,0.82}
\lstdefinestyle{mystyle}{
    commentstyle=\color{codegreen},
    keywordstyle=\color{magenta},
    numberstyle=\tiny\color{codegray},
    stringstyle=\color{codepurple},
    basicstyle=\ttfamily\footnotesize,
    breakatwhitespace=false,         
    breaklines=true,                 
    captionpos=b,                    
    keepspaces=true,                 
    numbers=left,                    
    numbersep=5pt,                  
    showspaces=false,                
    showstringspaces=false,
    showtabs=false,                  
    tabsize=2
}

\lstset{style=mystyle}
%\vdash
\begin{document}
\author{Yosef Goren \& Andrew }
\title{Intro to Software Verification - Homework 3}
\maketitle
\section*{Question 1}
%enumerate with Capital letters A,B,...
\begin{enumerate}[label=\Alph*.]
    \item 3
    \item 1
    \item 3
    \item 2
    \item 1
\end{enumerate}


\section*{Question 2}
\begin{enumerate}
    \item True. Let $\pi=s_0\rightarrow s_5\rightarrow s_5$.
        \begin{itemize}
            \item $M,\pi^2\models b$
            \item $M,\pi^1\models Xb$
            \item $M,\pi^0\models XXb$
            \item $M\models E[XXb]$
        \end{itemize}
    \item True. Let $\pi$ be an arbitrary path in $M$.\\
        $\pi$ must be in the form $s_0\rightarrow v\rightarrow *$
        where $v\in\{s_1,s_4,s_5\}$.\\
        We want to prove $M,\pi\models (EXa)U(EXc)$.\\
        \[\left((s_0,s_1)\in M\right)\wedge\left(s_1\models a\right)\Rightarrow s_0\models EXa\Rightarrow \pi^0\models EXa\]
        Additionally:
        \[\forall u\in\{s_1,s_4,s_5\}, \exists u':(u,u')\in M\wedge u'\models c\]
        \[\Rightarrow \forall u\in\{s_1,s_4,s_5\}, u\models EXc\]
        \[\Rightarrow v\models EXc\Rightarrow \pi^1\models EXc\]
        \[\Rightarrow M,\pi\models (EXa)U(EXc)\]
    \item True. Let $\pi=s_0\rightarrow s_4\rightarrow s_7$.
        \begin{itemize}
            \item $M,\pi^2\models Gc$
            \item $M,\pi^1\models a$
            \item $M,\pi^1\models aU(Gc)$
            \item $M,\pi^0\models b$
            \item $M,\pi^0\models bU(U(Gc))$
            \item $M\models E[bU(U(Gc))]$
        \end{itemize}
    \item True. Let $\pi=s_0\rightarrow *$.\\
        \begin{itemize}
            \item $s_0\models b$
            \item $s_0\models cUb$
            \item $s_0\models a(cUb)$
            \item $\pi\models a(cUb)$
        \end{itemize}
        $\Rightarrow M\models A[a(cUb)]$
\end{enumerate}
\section*{Question 3}
\subsection*{Part A.}
Let: $H:=\{f_1\times f_2\times, ...,\times f_m\}$.\\
In other words, $H$ is the set of all possible combinations of the $m$ functions in $F$.\\
For any $h\in H, i\in[m]$, let $h[i]$ be an item within $h$
which was chosen from $f_i$ - one must exists since $h$ is a combination of $f_i$.\\
More formally, let $h[i]:= argmin_{i\mid s_i\in h\cap f_i}$ (the minimal item in $h$ from $f_i$).\\

Proof.\\
The following logical formulas are equivalent (and the transition from one to the other is trivial):
\begin{enumerate}
    \item $\forall i\in[m], f_i\cap inf(\pi)\neq\emptyset$
    \item $\forall i\in[m],\exists s_i, s_i\in f_i\wedge s_i\in inf(\pi)$
    \item $\forall i\in[m],\exists s_i\in f_i, s_i\in inf(\pi)$
    \item $\exists s_1,s_2,...,s_m,\forall i\in[m], s_i\in f_i\wedge s_i\in inf(\pi)$
    \item $\exists h\in H,\forall i\in[m], h[i]\in f_i\wedge h[i]\in inf(\pi)$
    \item $\exists h\in H,(\forall i\in[m], h[i]\in f_i)\wedge (\forall i\in[m], h[i]\in inf(\pi))$
    \item $\exists h\in H,(h\subseteq inf(\pi))\wedge (\forall i\in[m], h[i]\in inf(\pi))$
    \item $\exists h\in H,(h\subseteq inf(\pi))\wedge (true)$
    \item $\exists h\in H, h\subseteq inf(\pi)$
\end{enumerate}

\subsection*{Part B.}
For any $i\in[m], \bar{f_i}:=S\setminus f_i$.\\
Let $h:=\bigcap_{i=1}^m\bar{f_i}, H:=\{h\}$.\\

Proof.\\
The following series of formulas are equivalent:
\begin{enumerate}
    \item $\forall i\in[m], f_i\cap inf(\pi)=\emptyset$
    \item $\forall i\in[m], \bar{f_i}\cap inf(\pi)=inf(\pi)$
    \item $\forall i\in[m], inf(\pi)\subseteq \bar{f_i}$
    \item $\forall i\in[m], \forall s\in inf(\pi),s\in\bar{f_i}$
    \item $\forall s\in inf(\pi), \forall i\in[m],s\in\bar{f_i}$
    \item $\forall s\in inf(\pi), s\in\bigcap_{i=1}^m\bar{f_i}$
    \item $inf(\pi)\subseteq bigcap_{i=1}^m\bar{f_i}$
    \item $inf(\pi)\subseteq h$
    \item $\forall h'\in H, inf(\pi)\subseteq h'$
\end{enumerate}

\section*{Question 4}
\subsection*{Part A.}
Let $\phi_{B\rightarrow W}:=b\wedge bU(Gw)$,\\
Let $\phi_{W\rightarrow B}:=w\wedge wU(Gb)$.\\
The meaning of $\phi_{B\rightarrow W}$ is that the 
path starts from at-least one $b$, and then continues
being $b$ right untill the point where it is $Gw$, meaning it
is $w$ exclusively from forever. In other words - 
there is exactly one transision from $b$ to $w$ and no
transisions from $w$ to $b$.\\
Symmetrically, $\phi_{W\rightarrow B}$ means there is
exactly one transision from $w$ to $b$ and no other transisions.\\
This means $\phi_{W\rightarrow B}\wedge \phi_{B\rightarrow W}$ is a 
satisfactory and required for the path to be legal.\\
Let $\phi_L:=\phi_{W\rightarrow B}\wedge \phi_{B\rightarrow W}$.\\
So $\phi_L$ means that the path is legal.\\
\[
    \phi_L=(b\wedge bU(Gw))\wedge (w\wedge wU(Gb))    
\]
And finally, we want to require there is a legal path so define:
$\phi:E\phi_L$.\\
\[
    \phi=E((b\wedge bU(Gw))\wedge (w\wedge wU(Gb)))
\]

\end{document}