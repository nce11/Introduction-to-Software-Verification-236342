\documentclass{article}
% basics
\usepackage{amsfonts}
\usepackage{enumitem}
\usepackage{float}
\usepackage{graphicx}
\usepackage{hyperref} 
\usepackage[labelfont=bf]{caption}

\newtheorem{theorem}{Theorem}
\newtheorem{lemma}[theorem]{Lemma}
\newtheorem{corollary}{Corollary}[theorem]

% unique math expressions:  
\usepackage{amsmath}
\DeclareMathOperator*{\andloop}{\wedge}
\DeclareMathOperator*{\pr}{Pr}
\DeclareMathOperator*{\approach}{\longrightarrow}
\DeclareMathOperator*{\eq}{=}

% grey paper
\usepackage{xcolor}
% \pagecolor[rgb]{0.11,0.11,0.11}
% \color{white}

% embedded code sections
\usepackage{listings}
\definecolor{codegreen}{rgb}{0,0.6,0}
\definecolor{codegray}{rgb}{0.5,0.5,0.5}
\definecolor{codepurple}{rgb}{0.58,0,0.82}
\lstdefinestyle{mystyle}{
    commentstyle=\color{codegreen},
    keywordstyle=\color{magenta},
    numberstyle=\tiny\color{codegray},
    stringstyle=\color{codepurple},
    basicstyle=\ttfamily\footnotesize,
    breakatwhitespace=false,         
    breaklines=true,                 
    captionpos=b,                    
    keepspaces=true,                 
    numbers=left,                    
    numbersep=5pt,                  
    showspaces=false,                
    showstringspaces=false,
    showtabs=false,                  
    tabsize=2
}

\lstset{style=mystyle}

\begin{document}
\author{Yosef Goren, Andrew Elashkin}
\title{Introduction to Software Verification 236342, Homework 1}
\maketitle
\section{}
%enumerate with capital letters:
\begin{enumerate}[label=\Alph*.]
    \item Correct. Since the precondition is false, the postcondition is 'always' satisfied (since it is never tested).
    \item Incorrect. Counterexample $x=-100,y=-99$:
    \begin{itemize}
        \item $l_0, -100, -99$
        \item $l_1, -100, -99$
        \item $l_2, -100, -99$
        \item $l_3, -1, -99$
        \item $l_*, -1, -99$
    \end{itemize}
    As can be seen, precondition is satisfied and postcondition is not.
    \item Correct. The postcondition is true, so regardless of anything else,
    for every input selection it will be evaluated as true (the program does not even have to terminate either).
    \item Incorrect. Counterexample $x=1, y=9$:
    \begin{itemize}
        \item $l_0, 1, 9$
        \item $l_1, 1, 9$
        \item $l_2, -8, 9$
        \item $l_3, -8, 9$
        \item $l_*, -8, 9$
    \end{itemize}
    Postcondition is false, so it is not satisfied.
    \item Incorrect. Counterexample $x=1, y=3$:
    \begin{itemize}
        \item $l_0, 1, 3$
        \item $l_1, 1, 3$
        \item $l_2, 1, 3$
        \item $l_3, -2, 3$
        \item $l_4, -2, 3$
        \item $l_1, -2, -3$
        \item $l_2, -2, -3$
        \item $l_3, -2, -3$
        \item $l_4, -2, -3$
        \item $l_1, -2, 3$
        \item $l_2, -2, 3$
        \item $l_3, -5, 3$
        \item $l_4, -5, 3$
    \end{itemize}
    By running this example we see that the program gets stuck in a loop
    at labels $l_1, l_2, l_3, l_4$. And each 4 iterations result with the state
    being the same as the initial state at $l_1$.
    Since this is a total correctness condition on the specification,
    the correctness is contradicted by the program failing to terminate.
    \item Correct. We now require partial correctness, so we check postcondition only for the cases that terminate. That means, that we won't have a problem like we had in case E, since such an input won't be considered, and all the cases that satisfy $x^2<y^2$ and terminate hold $x^2<y^2$ postcondition.
    \item Incorrect. Since we require total correctness, every computation that satisfies the precondition $z=5$ should terminate. However, for $x=1, y=2, z=5$ the program never terminates, as we showed in E.
    \item Correct. The program does not change the value of $z$, so every run that terminates and satisfies the precondition $z=5$ will also hold the postcondition.
    \item Correct. The value of $|y|$ never changes during the execution, so the condition will hold for every run that terminates.
    \item Incorrect (But correct for $x,y \in \mathbb{N}$). If we assume that the input can be rational numbers, then $x=7.5, y=7.5$ will fail the postcondition. However, the assumption holds for the integers.
    \item Incorrect. $x=1, y=2$ satisfy the precondition, but the program never terminates, as we showed in E.
    \item Correct. We check for partial correctness, so we want the program that satisfies the precondition to stuck in infinite loop. This is exactly what happens with all the numbers satisfying the precondition, similar to the example in E.
\end{enumerate}

\section{}
\begin{enumerate}[label=\Alph*.]
    \item Specification is described by the tuple $<q_1, q_2>$ \\
    Where precondition $q_1$ is $$r=T, r \in \mathbb{N}_0, k \in \mathbb{N}_0$$
    Postcondition $q_2$ is: $$k \in \mathbb{N}_0 \wedge (r=1 \wedge (T>f_k) \wedge \mathcal{F}_k) \vee (r=0 \wedge (T=f_k) \wedge \mathcal{F}_k) \vee (r=-1 \wedge (T<F_k) \wedge \mathcal{F}_k)$$
    $$\mathcal{F}_k = \exists f_0, f_1,...,f_k(f_0=0 \wedge f_1=1 \wedge ( \bigwedge\limits_{l=2}^k f_l=f_{l-1} + f_{l-2} ) $$
    \item Specification is described by the tuple $<q_1, q_2>$ \\
    Precondition $q_1$ is $$x=X, x \in \mathbb{N}_0$$
    Postcondition $q_2$ is:
    $$ \exists Y \in \mathbb{N}_0 \wedge \exists r \in \mathbb{N}_0 \wedge (k \in \mathbb{N}_0 \wedge Fib_k(r)) \wedge \\
    Y*r\leq x \wedge (Y+1)*r > x \wedge \\
    (y=(Y+1)*r \wedge x=X)$$
\end{enumerate}

\section{}
\begin{enumerate}[label=\Alph*.]
    \item To enforce the program to not finish
    on a specific set on inputs, we can require that if
    said inputs have been given and the program finishes - 
    the postcondition fails:
    \[\{\forall p\in\mathbb{P}, x\neq p^2\}P\{false\}\]
    We can also represent $p\in\mathbb{P}$ more explicitly:
    \[p\in\mathbb{P}\Leftrightarrow
    (\forall x,\forall y, (x\cdot y=p)\rightarrow (|x|=1\wedge|y|=1))\]
    (Here we assume the variables are defined over the integers).\\
    So our full specification is:
    \[\{\forall p,
    ((\forall x,\forall y, (x\cdot y=p)\rightarrow (|x|=1)\wedge|y|=1)
    \rightarrow
    (x\neq p^2))\}P\{false\}\]
    \item To require that for a set of inputs a program finishes.
    we can use the precondition to apply the condition only to
    the relivant set and use total correctness to require the program to halt
    on these inputs:
    \[<gcd(x,y)=1>P<true>\]
    Similarly, we can express $gcd(x,y)=1$ with:
    \[
        (gcd(x,y)=1)
    \Leftrightarrow
        (\forall z,(\exists n, n\cdot z=x)\rightarrow (\forall n, n\cdot z\neq y))
    \]
    And get the full specification:
    \[<\forall x,\forall y,(\forall z,(\exists n, n\cdot z=x)\rightarrow (\forall n, n\cdot z\neq y))>P<true>\]
\end{enumerate}

\section{}
\begin{itemize}
    \item The reachability condition:
    \[
        R'^0(\bar{x})=true
    \]
    \[
        R'^{k+1}(\bar{x})=
        \begin{cases}
            R'^k(\bar{x}) & \text{if } l_{i_k}\in\{[start], [end], [\bar{y}:=\bar{e}]\}\\
            R'^k(\bar{x})\wedge B(\bar{x}) & \text{if } l_{i_k}=[B(\bar{X})]\wedge (l_{i_k}\rightarrow^Tl_{i_{k+1}})\\
            R'^k(\bar{x})\wedge \neg B(\bar{x}) & \text{if } l_{i_k}=[B(\bar{X})]\wedge (l_{i_k}\rightarrow^Fl_{i_{k+1}})

        \end{cases}    
    \]
    \item The state transformer:
    \[
        T'^0(\bar{x})=\bar{x}    
    \]
    \[
        T'^{k+1}(\bar{x})=
        \begin{cases}
            T'^k(\bar{x}) & \text{if } l_{i_k}\in\{[start], [end], [B(\bar{x})]\}\\
            T'^k(\bar{x})[\bar{y}\leftarrow\bar{e}] & \text{if } l_{i_k}=[\bar{y}:=\bar{e}]
        \end{cases}    
    \]
\end{itemize}


\end{document}
