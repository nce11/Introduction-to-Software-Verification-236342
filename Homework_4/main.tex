\documentclass{article}
% basics
\usepackage{amsfonts}
\usepackage{enumitem}
\usepackage{float}
\usepackage{graphicx}
\usepackage{hyperref} 
\usepackage[labelfont=bf]{caption}

\newtheorem{theorem}{Theorem}
\newtheorem{lemma}[theorem]{Lemma}
\newtheorem{corollary}{Corollary}[theorem]

% unique math expressions:  
\usepackage{amsmath}
\DeclareMathOperator*{\andloop}{\wedge}
\DeclareMathOperator*{\pr}{Pr}
\DeclareMathOperator*{\approach}{\longrightarrow}
\DeclareMathOperator*{\eq}{=}

% grey paper
\usepackage{xcolor}
% \pagecolor[rgb]{0.11,0.11,0.11}
% \color{white}

% embedded code sections
\usepackage{listings}
\definecolor{codegreen}{rgb}{0,0.6,0}
\definecolor{codegray}{rgb}{0.5,0.5,0.5}
\definecolor{codepurple}{rgb}{0.58,0,0.82}
\lstdefinestyle{mystyle}{
    commentstyle=\color{codegreen},
    keywordstyle=\color{magenta},
    numberstyle=\tiny\color{codegray},
    stringstyle=\color{codepurple},
    basicstyle=\ttfamily\footnotesize,
    breakatwhitespace=false,         
    breaklines=true,                 
    captionpos=b,                    
    keepspaces=true,                 
    numbers=left,                    
    numbersep=5pt,                  
    showspaces=false,                
    showstringspaces=false,
    showtabs=false,                  
    tabsize=2
}

\newcommand{\lv}[0]{\overline{v}}
\newcommand{\lx}[0]{\overline{x}}

\lstset{style=mystyle}

\begin{document}
\author{Yosef Goren}
\title{Software Verification Homework 4}
\section{BDD constructions}
%enumerate with letters:
\begin{enumerate}[label=\textbf{\alph*.}]
    \item 
        Denote the set of vertices: $V=\{\lx_i\mid i\in[n]\}$\\
        Let $E(\lv):=E_0(\lv)\wedge E_1(\lv)$.
        \[
            A'(\lv):=A(\lv)\wedge \left(\bigwedge_{i=0}^n(E(\lv,\lx_i)\Rightarrow B(\lx_i))\right)
        \]
        The idea is that $A(\lv)$ means the 'accepted' node has to be from $A$,
        and rest of the expression means that all of it's neighbors have to be in $B$.
        It is equivalent to satisfying the following formula:
        \[
            (\lv\in V)\wedge \left(\forall \lx\in V, E(\lx, \lv)\rightarrow B(\lx)\right)
        \]
    \item 
    \[
        V_{1,2}(\lv,\lv'):=
            \left(\bigvee_{i=0}^nE_1(\lv, \lx_i)\wedge E_0(\lx_i, \lv')\right)
            \vee
            \left(\bigvee_{i=0}^nE_0(\lv, \lx_i)\wedge E_1(\lx_i, \lv')\right)
    \]
    For the vertices $\lv,\lv'$ to have a 
    path of length 2 and weight 1 between them,
    there must either be a path of length 2 with weight
    1 were the edge connected to $\lv$ is 1 and the other edge is 0,
    or the other way around.\\
    The primary operator of the expression above describes this fact;
    to be more specific - the left side of the expression
    describes the case where there is a path of length 2 
    were the node connected to $\lv$ has weight 1, and so on.
\end{enumerate}
    
\section{BDD operations}
\section{D\&D}
\maketitle

\end{document}